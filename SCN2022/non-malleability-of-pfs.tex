% !TEX root = main.tex
% !TEX spellcheck = en-US

\section{Non-malleability of Plonk} 
\label{sec:plonk}
In this section, we show that $\plonkprotfs$ is simulation-extractable. To this end, we first use the unique opening property to show that
$\plonkprotfs$ has the $\ur{3}$ property,
cf.~\cref{lem:plonkprot_ur}.
Next, we show that $\plonkprotfs$ is rewinding-based knowledge sound. That is, given a number of accepting transcripts whose first $3$ messages match, we can either extract a witness for the proven statement or use one of the transcripts to break the $\udlog$ assumption. This result is shown in the AGM, cf.~\cref{lem:plonkprot_ss}. We then show that $\plonkprotfs$ is $3$-programmable trapdoor-less ZK in the AGM, cf.~\cref{lem:plonk_tlzk}.

Given rewinding-based knowledge soundness, $\ur{3}$ and trapdoor-less zero-knowledge of $\plonkprotfs$, we invoke \cref{thm:se} and conclude that $\plonkprotfs$ is simulation-extractable.

\newcommand{\vql}{\vec{q_{L}}}
\newcommand{\vqr}{\vec{q_{R}}}
\newcommand{\vqm}{\vec{q_{M}}}
\newcommand{\vqo}{\vec{q_{O}}}
\newcommand{\vx}{\vec{x}}
\newcommand{\vqc}{\vec{q_{C}}}

\subsection{Plonk Protocol Description}
\label{sec:plonk_explained}
\oursubsub{The constraint system.}
Assume $\CRKT$ is a fan-in two arithmetic circuit, whose
fan-out is unlimited and has $\numberofconstrains$ gates and $\noofw$ wires
($\numberofconstrains \leq \noofw \leq 2\numberofconstrains$). The constraint
system of $\plonk$ is defined as follows:
\begin{compactitem}
\item Let $\vec{V} = (\va, \vb, \vc)$, where $\va, \vb, \vc
  \in \range{1}{\noofw}^\numberofconstrains$. Entries $\va_i, \vb_i, \vc_i$ represent indices of left,
  right and output wires of the circuit's $i$-th gate.
\item Vectors $\vec{Q} = (\vql, \vqr, \vqo, \vqm, \vqc) \in
  (\FF^\numberofconstrains)^5$ are called \emph{selector vectors}:
  \begin{inparaenum}[(a)]
  \item If the $i$-th gate is a multiplication gate then $\vql_i = \vqr_i = 0$,
    $\vqm_i = 1$, and $\vqo_i = -1$. 
  \item If the $i$-th gate is an addition gate then $\vql_i = \vqr_i  = 1$, $\vqm_i =
    0$, and $\vqo_i = -1$. 
  \item $\vqc_i = 0$ for multiplication and addition gates.\footnote{The $\vqc_i$ selector vector is meant to encode (input independent) constants.} 
  \end{inparaenum}
\end{compactitem}

We say that vector $\vx \in \FF^\noofw$ satisfies constraint system if for all $i
\in \range{1}{\numberofconstrains}$
\[
  \vql_i \cdot \vx_{\va_i} + \vqr_i \cdot \vx_{\vb_i} + \vqo \cdot \vx_{\vc_i} +
  \vqm_i \cdot (\vx_{\va_i} \vx_{\vb_i}) + \vqc_i = 0. 
\]

Public inputs $\brak{\inp_j}_{j = 1}^{\instsize}$ are enforced by adding the constrains
\[ \va_i = j, \vql_i = 1, \vqm_i = \vqr_i = \vqo_i = 0, \vqc_i = -\inp_j\,,
\]
for some $i \in \range{1}{\noofc}$.

\oursubsub{Algorithms rolled out}
\label{sec:plonk_explained}
\plonk{} argument system is universal. That is, it allows to verify computation
of any arithmetic circuit which has up to $\numberofconstrains$
gates using a single SRS. However, to make computation efficient, for each
circuit there is a preprocessing phase which extends the SRS with
circuit-related polynomial evaluations.

For the sake of simplicity of the security reductions presented in this paper, we
include in the SRS only these elements that cannot be computed without knowing
the secret trapdoor $\chi$. The rest of the preprocessed input can
be computed using these SRS elements. We thus let them to be computed by the
prover, verifier, and simulator separately.

\ourpar{$\plonk$ SRS generating algorithm $\kgen(\REL)$:}
The SRS generating algorithm picks at random $\chi \sample \FF_p$, computes
and outputs
\(
	\srs = \left(\gone{\smallset{\chi^i}_{i = 0}^{\numberofconstrains + 5}},
	\gtwo{\chi} \right).
\)

\ourpar{Preprocessing:}
Let $H = \smallset{\omega^i}_{i = 1}^{\numberofconstrains }$ be a
(multiplicative) $\numberofconstrains$-element subgroup of a field $\FF$
compound of $\numberofconstrains$-th roots of unity in $\FF$. Let $\lag_i(X)$ be
the $i$-th element of an $\numberofconstrains$-elements Lagrange basis. During
the preprocessing phase polynomials $\p{S_{id j}}, \p{S_{\sigma j}}$, for
$\p{j} \in \range{1}{3}$, are computed:
\begin{equation*}
  \begin{aligned}
    \p{S_{id 1}}(X) & = X,\\[\myskip]
    \p{S_{id 2}}(X) & = k_1 \cdot X,\\[\myskip]
    \p{S_{id 3}}(X) & = k_2 \cdot X,
  \end{aligned}
  \qquad
\begin{aligned}
  \p{S_{\sigma 1}}(X) & = {\textstyle{\sum_{i = 1}^{\noofc} \sigma(i) \lag_i(X)}},\\[\myskip]
  \p{S_{\sigma 2}}(X) & = {\textstyle \sum_{i = 1}^{\noofc}
  \sigma(\noofc + i) \lag_i(X)},\\[\myskip]
  \p{S_{\sigma 3}}(X) & ={\textstyle\sum_{i = 1}^{\noofc} \sigma(2 \noofc + i) \lag_i(X)}.
\end{aligned}
\end{equation*}
Coefficients $k_1$, $k_2$ are such that $H, k_1 \cdot H, k_2 \cdot H$ are
different cosets of $\FF^*$, thus they define $3 \cdot \noofc$
different elements. Gabizon et al.~\cite{EPRINT:GabWilCio19} notes that it is enough to set
$k_1$ to a quadratic residue and $k_2$ to a quadratic non-residue.

Furthermore, we define polynomials $\p{q_L}, \p{q_R}, \p{q_O}, \p{q_M}, \p{q_C}$
such that
\begin{equation*}
  \begin{aligned}
  \p{q_L}(X) & = {\textstyle \sum_{i = 1}^{\noofc}} \vql_i \lag_i(X), \\
  \p{q_R}(X) & = \textstyle \sum_{i = 1}^{\noofc} \vqr_i \lag_i(X), \\
  \p{q_M}(X) & = \textstyle \sum_{i = 1}^{\noofc} \vqm_i \lag_i(X),
\end{aligned}
\qquad
\begin{aligned}
  \p{q_O}(X) & = \textstyle  \sum_{i = 1}^{\noofc} \vqo_i \lag_i(X), \\
  \p{q_C}(X) & =  \textstyle \sum_{i = 1}^{\noofc} \vqc_i \lag_i(X). \\
  \vphantom{\p{q_M}(X)  = \textstyle \sum_{i = 1}^{\noofc} \vqm_i \lag_i(X),}
\end{aligned}
\end{equation*}

\ourpar{Proving statements in $\plonkprotfs$} We show how prover's algorithm
$\prover(\srs, \inp=\brak{\wit'_i}_{i = 1}^\instsize, \wit = \brak{\wit_i}_{i=1}^{3 \cdot \noofc})$ operates for
the Fiat--Shamir transformed version of Plonk. Note that for notational convenience $\wit$ also contains the public input wires $\wit'_i=\wit_i$, $i\in \range{1}{\ell}$.
\begin{description}
\item[Message 1] Sample $b_1, \ldots, b_9 \sample \FF_p$; compute
  $\p{a}(X), \p{b}(X), \p{c}(X)$ as
	\begin{align*}
		\p{a}(X) &= (b_1 X + b_2)\p{Z_H}(X) + \textstyle \sum_{i = 1}^{\noofc} \wit_i \lag_i(X) \\
		\p{b}(X) &= (b_3 X + b_4)\p{Z_H}(X) + \textstyle \sum_{i = 1}^{\noofc} \wit_{\noofc + i} \lag_i(X) \\
		\p{c}(X) &= (b_5 X + b_6)\p{Z_H}(X) + \textstyle \sum_{i = 1}^{\noofc} \wit_{2 \cdot \noofc + i} \lag_i(X) 
	\end{align*}
	Output polynomial commitments $\gone{\p{a}(\chi), \p{b}(\chi), \p{c}(\chi)}$.  
	
\item[Message 2] Compute challenges $\beta, \gamma \in \FF_p$ by querying random oracle
  on partial proof, that is,
	\(
		\beta = \ro(\tzkproof[0..1], 0)\,, \qquad \gamma = \ro(\tzkproof[0..1], 1)\,.
	\)
  
	Compute permutation polynomial $\p{z}(X)$
	\begin{multline*}
		\p{z}(X) = (b_7 X^2 + b_8 X + b_9)\p{Z_H}(X) + \lag_1(X) + \\
    + \sum_{i = 1}^{\noofc - 1} \left(\lag_{i + 1} (X) \prod_{j = 1}^{i} \frac{
        (\wit_j +\beta \omega^{j - 1} + \gamma)(\wit_{\noofc + j} + \beta k_1
        \omega^{j - 1} + \gamma)(\wit_{2 \noofc + j} +\beta k_2 \omega^{j- 1} +
        \gamma)} {(\wit_j+\sigma(j) \beta + \gamma)(\wit_{\noofc + j} + \sigma(\noofc
        + j)\beta + \gamma)(\wit_{2 \noofc + j} + \sigma(2 \noofc + j)\beta +
        \gamma)}\right)
	\end{multline*}
	Output polynomial commitment $\gone{\p{z}(\chi)}$
		
\item[Message 3] Compute the challenge $\alpha = \ro(\tzkproof[0..2])$, compute the quotient
  polynomial
	\begin{align*}
    & \p{t}(X)  = \\
    & (\p{a}(X) \p{b}(X) \selmulti(X) + \p{a}(X) \selleft(X) + 
      \p{b}(X)\selright(X) + \p{c}(X)\seloutput(X) + \pubinppoly(X) + \selconst(X)) /  
      \p{Z_H}(X) +\\
    & + ((\p{a}(X) + \beta X + \gamma) (\p{b}(X) + \beta k_1 X + \gamma)(\p{c}(X) 
      + \beta k_2 X + \gamma)\p{z}(X)) \infrac{\alpha}{\p{Z_H}(X)} \\
    & - (\p{a}(X) + \beta \p{S_{\sigma 1}}(X) + \gamma)(\p{b}(X) + \beta 
      \p{S_{\sigma 2}}(X) + \gamma)(\p{c}(X) + \beta \p{S_{\sigma 3}}(X) + 
      \gamma)\p{z}(X \omega))  \infrac{\alpha}{\p{Z_H}(X)} \\
    & + (\p{z}(X) - 1) \lag_1(X) \infrac{\alpha^2}{\p{Z_H}(X)}
	\end{align*}
	Split $\p{t}(X)$ into degree less then $\noofc$ polynomials
  $\p{t_{lo}}(X), \p{t_{mid}}(X), \p{t_{hi}}(X)$, such that
	\(
		\p{t}(X) = \p{t_{lo}}(X) + X^{\noofc} \p{t_{mid}}(X) + X^{2 \noofc}
    \p{t_{hi}}(X)\,.
	\)
	Output $\gone{\p{t_{lo}}(\chi), \p{t_{mid}}(\chi), \p{t_{hi}}(\chi)}$.
	
\item[Message 4] Get the challenge $\chz \in \FF_p$, $\chz = \ro(\tzkproof[0..3])$.
  Compute opening evaluations
	\(
    \p{a}(\chz), \p{b}(\chz), \p{c}(\chz), \p{S_{\sigma 1}}(\chz), \p{S_{\sigma 2}}(\chz), \p{t}(\chz), \p{z}(\chz \omega),
	\)
	Compute the linearization polynomial
	\[
		\p{r}(X) =
		\begin{aligned}
      & \p{a}(\chz) \p{b}(\chz) \selmulti(X) + \p{a}(\chz) \selleft(X) + \p{b}(\chz) \selright(X) + \p{c}(\chz) \seloutput(X) + \selconst(X) \\
      & + \alpha \cdot \left( (\p{a}(\chz) + \beta \chz + \gamma) (\p{b}(\chz) + \beta k_1 \chz + \gamma)(\p{c}(\chz) + \beta k_2 \chz + \gamma) \cdot \p{z}(X)\right) \\
      & - \alpha \cdot \left( (\p{a}(\chz) + \beta \p{S_{\sigma 1}}(\chz) + \gamma) (\p{b}(\chz) + \beta \p{S_{\sigma 2}}(\chz) + \gamma)\beta \p{z}(\chz\omega) \cdot \p{S_{\sigma 3}}(X)\right) \\
      & + \alpha^2 \cdot \lag_1(\chz) \cdot \p{z}(X)
		\end{aligned}
	\]
	Output
  $\p{a}(\chz), \p{b}(\chz), \p{c}(\chz), \p{S_{\sigma 1}}(\chz), \p{S_{\sigma
      2}}(\chz), \p{t}(\chz), \p{z}(\chz \omega), \p{r}(\chz).$
	
\item[Message 5] Compute the opening challenge $v \in \FF_p$,
  $v = \ro(\tzkproof[0..4])$.  Compute the openings for the polynomial commitment
  scheme
	\hspace*{-2cm}\begin{align*}
	& \p{W_\chz}(X) = \frac{1}{X - \chz} \left(
   \begin{aligned}
     & \p{t_{lo}}(X) + \chz^\noofc \p{t_{mid}}(X) + \chz^{2 \noofc} \p{t_{hi}}(X) - \p{t}(\chz)
     + v(\p{r}(X) - \p{r}(\chz)) 
      + v^2 (\p{a}(X) - \p{a}(\chz))\\
     & + v^3 (\p{b}(X) - \p{b}(\chz))
      + v^4 (\p{c}(X) - \p{c}(\chz))
      + v^5 (\p{S_{\sigma 1}}(X) - \p{S_{\sigma 1}}(\chz)) \\
     & + v^6 (\p{S_{\sigma 2}}(X) - \p{S_{\sigma 2}}(\chz))
   \end{aligned}
       \right)\\
    & \p{W_{\chz \omega}}(X) = \infrac{(\p{z}(X) - \p{z}(\chz \omega))}{(X - \chz \omega)}
  \end{align*}
	Output $\gone{\p{W_{\chz}}(\chi), \p{W_{\chz \omega}}(\chi)}$.
\end{description}

\ncase{Plonk verifier $\verifier(\srs, \inp, \zkproof)$}\ \newline
The \plonk{} verifier works as follows
\begin{enumerate}
	\item Validate all obtained group elements.
	\item Validate all obtained field elements.
	\item Parse the instance as
      $\smallset{\wit_i}_{i = 1}^\instsize \gets \inp$.
	\item Compute challenges $\beta, \gamma, \alpha, \chz, v, u$ from the transcript.
	\item Compute zero polynomial evaluation
      $\p{Z_H} (\chz) =\chz^\noofc - 1$.
	\item Compute Lagrange polynomial evaluation
      $\lag_1 (\chz) = \frac{\chz^\noofc -1}{\noofc (\chz - 1)}$.
	\item Compute public input polynomial evaluation
      $\pubinppoly (\chz) = \sum_{i \in \range{1}{\instsize}} \wit_i
      \lag_i(\chz)$.
	\item Compute quotient polynomials evaluations
	\begin{multline*}
    \p{t} (\chz) =  \Big(
    \p{r} (\chz) + \pubinppoly(\chz) - (\p{a}(\chz) + \beta \p{S_{\sigma 1}}(\chz) + \gamma) (\p{b}(\chz) + \beta \p{S_{\sigma 2}}(\chz) + \gamma) 
    (\p{c}(\chz) + \gamma)\p{z}(\chz \omega) \alpha - \lag_1 (\chz) \alpha^2
    \Big) / {\p{Z_H}(\chz)} \,.
	\end{multline*}
	\item Compute batched polynomial commitment
	$\gone{D} = v \gone{r} + u \gone {z}$ that is
	\begin{align*}
		\gone{D} & = v
		\left(
		\begin{aligned}
          & \p{a}(\chz)\p{b}(\chz) \cdot \gone{\selmulti} + \p{a}(\chz)  \gone{\selleft} + \p{b}  \gone{\selright} + \p{c}  \gone{\seloutput} + \\
          & + (	(\p{a}(\chz) + \beta \chz + \gamma) (\p{b}(\chz) + \beta k_1 \chz + \gamma) (\p{c} + \beta k_2 \chz + \gamma) \alpha  + \lag_1(\chz) \alpha^2)  + \\
			% &   \\
          & - (\p{a}(\chz) + \beta \p{S_{\sigma 1}}(\chz) + \gamma) (\p{b}(\chz)
          + \beta \p{S_{\sigma 2}}(\chz) + \gamma) \alpha \beta \p{z}(\chz
          \omega) \gone{\p{S_{\sigma 3}}(\chi)})
		\end{aligned}
		\right) + \\
		& + u \gone{\p{z}(\chi)}\,.
	\end{align*}
	\item Computes full batched polynomial commitment $\gone{F}$:
	\begin{align*}
      \gone{F} & = \left(\gone{\p{t_{lo}}(\chi)} + \chz^\noofc \gone{\p{t_{mid}}(\chi)} + \chz^{2 \noofc} \gone{\p{t_{hi}}(\chi)}\right) + u \gone{\p{z}(\chi)} + \\
               & + v
                 \left(
		\begin{aligned}
			& \p{a}(\chz)\p{b}(\chz) \cdot \gone{\selmulti} + \p{a}(\chz)  \gone{\selleft} + \p{b}(\chz)   \gone{\selright} + \p{c}(\chz)  \gone{\seloutput} + \\
			& + (	(\p{a}(\chz) + \beta \chz + \gamma) (\p{b}(\chz) + \beta k_1 \chz + \gamma) (\p{c}(\chz)  + \beta k_2 \chz + \gamma) \alpha  + \lag_1(\chz) \alpha^2)  + \\
			% &   \\
			& - (\p{a}(\chz) + \beta \p{S_{\sigma 1}}(\chz) + \gamma) (\p{b}(\chz) + \beta \p{S_{\sigma 2}}(\chz) + \gamma) \alpha  \beta \p{z}(\chz \omega) \gone{\p{S_{\sigma 3}}(\chi)})
		\end{aligned}
		\right) \\
		& + v^2 \gone{\p{a}(\chi)} + v^3 \gone{\p{b}(\chi)} + v^4 \gone{\p{c}(\chi)} + v^5 \gone{\p{S_{\sigma 1}(\chi)}} + v^6 \gone{\p{S_{\sigma 2}}(\chi)}\,.
	\end{align*}
	\item Compute group-encoded batch evaluation $\gone{E}$
	\begin{align*}
		\gone{E}  = \frac{1}{\p{Z_H}(\chz)} & \gone{
		\begin{aligned}
			& \p{r}(\chz) + \pubinppoly(\chz) +  \alpha^2  \lag_1 (\chz) + \\
			& - \alpha \left( (\p{a}(\chz) + \beta \p{S_{\sigma 1}} (\chz) + \gamma) (\p{b}(\chz) + \beta \p{S_{\sigma 2}} (\chz) + \gamma) (\p{c}(\chz) + \gamma) \p{z}(\chz \omega) \right)
		\end{aligned}
           }\\
      + & \gone{v \p{r}(\chz) + v^2 \p{a}(\chz) + v^3 \p{b}(\chz) + v^4 \p{c}(\chz) + v^5 \p{S_{\sigma 1}}(\chz) + v^6 \p{S_{\sigma 2}}(\chz) + u \p{z}(\chz \omega) }\,.
	\end{align*}
\item Check whether the verification
 % $\vereq_\zkproof(\chi)$
  equation holds
	\begin{multline}
		\label{eq:ver_eq} 
		\left( \gone{\p{W_{\chz}}(\chi)} + u \cdot \gone{\p{W_{\chz
                \omega}}(\chi)} \right) \bullet
		\gtwo{\chi} - \\
		\left( \chz \cdot \gone{\p{W_{\chz}}(\chi)} + u \chz \omega \cdot
          \gone{\p{W_{\chz \omega}}(\chi)} + \gone{F} - \gone{E} \right) \bullet
        \gtwo{1} = 0\,.
	\end{multline}
  The verification equation is a batched version of the verification equation
  from \cite{AC:KatZavGol10} which allows the verifier to check openings of
  multiple polynomials in two points (instead of checking an opening of a single
  polynomial at one point).
\end{enumerate}

\ncase{Plonk simulator $\simulator_\chi(\srs, \td= \chi, \inp)$}\ 
We describe the simulator in \cref{lem:plonk_tlzk}.

\subsection{Simulation extractability of $\plonk${}}
Due to lack of space, we provide here only theorem statements and intuition for why they hold. Full proofs are given in \cref{sec:plonkse_proofs}.

\oursubsub{Unique Response Property}
\begin{lemma}
	\label{lem:plonkprot_ur}
	Let $\PCOMp$ be a polynomial commitment that is $\epsbind(\secpar)$-binding and has unique opening property with loss $\epsop (\secpar)$. Then $\plonkprotfs$ is $\ur{3}$ against algebraic adversaries, who makes up to $q$ random oracle queries, with security loss $\epsbinding (\secpar) + \epsop ( \secpar )$.
\end{lemma}
\begin{proof}[Intuition]
  We show that an adversary who can break the $3$-unique response property of $\plonkprotfs$ can be either used to break the commitment scheme's evaluation binding or unique opening property. The former happens with the probability upper-bounded by $\epsbinding (\secpar)$, the latter with the probability upper bounded by $\epsop (\secpar)$. 
  
  %We note that the probability of breaking evaluation binding increases by an additional $q / p$ (at most) since $\plonk$ batches verification of polynomial evaluations, which may lead to accepting an opening which would not be accepted by a verifier who verifies all openings separately. Similarly, the additional $q / p$ security loss from unique opening property comes from the batching as well.
\end{proof}

\oursubsub{Rewinding-Based Knowledge Soundness}
\begin{lemma}
	\label{lem:plonkprot_ss}
  $\plonkprotfs$ is $(3, 3 \noofc + 6)$-rewinding-based knowledge sound against algebraic adversaries who make up to $q$ random oracle queries with security loss 
  \[
    \epscss(\secpar,\accProb, q) \leq \left(1 - \frac{\accProb - (q + 1) \left(\frac{3 \noofc + 5}{p} \right)}{1 - \frac{3 \noofc + 5}{p}}\right) + (3 \noofc + 6) \cdot \epsudlog (\secpar) %+ (3 \noofc + 6) \cdot \epsid (\secpar)
    \,,
  \]
	Here $\accProb$ is a probability that the adversary outputs an accepting proof, 
  %$\epsid(\secpar)$ is a soundness error of the ideal verifier for $\plonkprot$, 
  and $\epsudlog(\secpar)$ is security of $(\numberofconstrains + 5, 1)$-$\udlog$ 
	assumption.
\end{lemma}

\begin{proof}[Intuition]
  We use Attema et al.~\cite[Proposition 2]{EPRINT:AttFehKlo21} to bound the probability that an algorithm $\tdv$ does not obtain a tree of accepting transcripts in an expected number of runs. This happens with probability at most
  \[
    1 - \frac{\accProb - (q + 1) \left(\frac{3 \noofc + 5}{p} \right)}{1 - \frac{3 \noofc + 5}{p}}
  \]
  Then we analyze the case that one of the proofs in the tree $\tree$ outputted by $\tdv$ is not accepting by the ideal verifier. This discrepancy can be used to break an instance of an updatable dlog assumption which happens with probability at most $(3 \noofc + 6)  \cdot \epsudlog (\secpar)$. %Additionally, it may be impossible to extract the witness from a tree where each of the transcripts is accepting by the ideal verifier if the adversary broke soundness of the ideal verifier in one of the transcripts. That happens with probability at most $(3 \noofc + 6) \cdot \epsid (\secpar)$.
 \end{proof}
  

\oursubsub{Trapdoor-Less Zero-Knowledge of Plonk}
\begin{lemma}
  \label{lem:plonk_tlzk}
  $\plonkprotfs$ is 3-programmable trapdoor-less zero-knowledge.
\end{lemma}

\begin{proof}[Intuition]
  The simulator, that does not know the SRS trapdoor can make a simulated proof by programming the random oracle. It proceeds as follows. It picks a random witness and behaves as an honest prover up to the point when a commitment to the polynomial $\p{t}(X)$ is sent. Since the simulator picked a random witness and $\p{t}(X)$ is a polynomial only (modulo some negligible function) when the witness is correct, it cannot compute commitment to $\p{t}(X)$ as it is a rational function. However, the simulator can pick a random challenge $\chz$ and a polynomial $\p{\tilde{t}}(X)$ such that $\p{t} (\chz)  = \p{\tilde{t}} (\chz)$. Then the simulator continues behaving as an honest prover. We argue that such a simulated proof is indistinguishable from a real one.
\end{proof}

\subsection*{Simulation Extractability of~$\plonkprotfs$}
Since \cref{lem:plonkprot_ur,lem:plonkprot_ss,lem:plonk_tlzk} hold, $\plonkprot$ is $\ur{3}$,
rewinding-based knowledge sound and trapdoor-less zero-knowledge. We now make use of \cref{thm:se} and show that
$\plonkprot_\fs$ is simulation-extractable as defined in \cref{def:updsimext}.

\begin{corollary}[Simulation extractability of $\plonkprot_\fs$]
\label{thm:plonkprotfs_se}
$\plonkprotfs$ is \emph{updatable simulation-extractable} against any $\ppt$ adversary $\advse$ who makes up to $q$ random oracle queries and returns an accepting proof with probability at least $\accProb$ with extraction failure probability 
\[
  \epsse(\secpar, \accProb, q) \leq \left(1 - \frac{\accProb - \epsur (\secpar) - (q + 1) \epserr (\secpar)} {1 - \epserr (\secpar)}\right) + (3 \noofc + 6) \cdot \epsudlog (\secpar)\COMMENT{ + (3 \noofc + 6) \cdot \epsid (\secpar)},
\]
where $\epserr (\secpar) = \frac{3 \noofc + 5}{p}$, $\epsur (\secpar) \leq \epsbind (\secpar) + \epsop (\secpar)$, $p$ is the size of the field, and $\noofc$ is the number of constrains in the circuit. 
\end{corollary}
 
%%% Local Variables:
%%% mode: latex
%%% TeX-master: "main"
%%% End:
