\begin{figure}[t!]
	\centering
	% \small
	\centerline{\fbox{
		\begin{minipage}[t]{0.75\linewidth}
		\begin{pcvstack}
		\begin{pchstack}
			\procedure{$\kgen(\secparam, \maxdeg)$}{
				\chi \sample \FF_p \\ [\myskip]
				\srs := 
				\left( \gone{\smallset{\chi^i}_{i = 0}^{\maxdeg}},
				\gtwo{\chi} \right); \\
				\rho =  \left(\gone{\chi, \chi}, \gtwo{\chi}\right) \\ [\myskip]
				\pcreturn (\srs, \rho) \\ [\myskip]
			}
		%
		\pchspace
		% 
		\procedure{$\upd(\srs, \{\rho_j \}_{j=1}^n)$}{
			\text{Parse } \srs \text{ as } \left( \gone{\smallset{A_i}_{i = 0}^{\maxdeg}},
			\gtwo{B} \right) \\ [\myskip]
			\chi' \sample \FF_p  \\ [\myskip]
			\srs' := 
			\left( \gone{\smallset{{\chi'}^i A_i}_{i = 0}^{\maxdeg}},
			\gtwo{\chi' B} \right); \\
			\rho' =	\left( \gone{\chi' A_1, \chi'}, \gtwo{\chi'}\right) \\ [\myskip]
			\pcreturn (\srs', \rho')
		}
		\end{pchstack}
		\pcvspace
		\begin{pchstack}
		%
		\procedure{$\verifyCRS(\srs, \{\rho_j \}_{j=1}^n)$}{
			\text{Parse }  \srs \text{ as } \left( \gone{\smallset{A_i}_{i = 0}^{\maxdeg}},
			\gtwo{B} \right) \\ \text{and } \{\rho_j \}_{j=1}^n \text{ as } \left\{\left( P_j, \bP_j, \hP_j \right)\right\}_{j=1}^n \\ [\myskip]
			\text{Verify Update proofs: } \\ [\myskip]
			\t \bP_1 = P_1 \\ [\myskip]
			\t P_j \bullet \gtwo{1} = P_{j-1} \bullet \hP_j \quad \forall j \geq 2 \\ [\myskip]
			\t \bP_n \bullet \gtwo{1} = \gone{1} \bullet \hP_n \\ [\myskip]
			\text{Verify SRS structure: } \\ [\myskip]
			\t \gone{A_i} \bullet \gtwo{1} = \gone{A_{i-1}} \bullet \gtwo{B} \text{ for all } 0 < i \leq \maxdeg %\\ [\myskip]
		}
		%
		\end{pchstack}
		\end{pcvstack}
		%
		\end{minipage}}}
	\caption{Updatable SRS scheme $\SRScer$ for $\PCOMp$} 
	\label{fig:upd-scheme}
\end{figure}


\begin{figure}
  \small
  \centerline{\fbox{
\begin{minipage}{14.3cm}
\begin{pcvstack}[]
  \begin{pchstack}
			\procedure{$\SRScer(\secparam, \maxdeg)$}
			{
			\text{cf.~\cref{fig:upd-scheme}}
      }\\
      
			\hspace*{1.7cm}
			\pchspace
			
			\procedure{$\com(\srs, \vec{\p{f}}(X))$}
			{ 
				\pcreturn \gone{\vec{c}} = \gone{\vec{\p{f}}(\chi)}\\ [\myskip]
        \fbox{$\pcreturn \vec{\p{f}} (X)$}\\
			}

      \pchspace
	  \hspace*{1.7cm}

      \procedure{$\open(\srs, \vec{z}, \vec{s}, \vec{\p{f}}(X), \aux_0)$}
			{
      \vec{\gamma} \gets \ro (g_0( \vec{z}, \vec{s}, \gone{\vec{c}}, \aux_0))\\[\myskip]
			\pcfor i \in \range{1}{\abs{\vec{z}}} \pcdo\\ [\myskip]
      \pcind \p{o}_j(X) \gets \sum_{i \in K_j} \gamma_{j}^{i - 1}
      \frac{\p{f}_{i}(X) - \p{f}_{i}(z_j)}{X - z_j}\\ [\myskip] 
      \pcreturn \vec{o} = \gone{\vec{\p{o}}(\chi)}\\ [\myskip]
      \fbox{$\pcreturn \vec{\p{o}} (X)$}
				% \hphantom{\hspace*{5.5cm}}	
			}

    \end{pchstack}
		 \pcvspace
    
		\begin{pchstack}
			\procedure{$\verifyb(\srs, \gone{\vec{c}}, \vec{z}, \vec{s}, \gone{\p{o}(\chi)}, (\aux_0,\aux_1))$}
			{
        \vec{\gamma} \gets \ro (g_0( \vec{z}, \vec{s}, \gone{\vec{c}}, \aux_0))\\[\myskip]
				r \gets \ro (g_1(\gone{\vec{c}}, \vec{z}, \vec{s}, \gone{\p{o}(\chi)}, \aux_1))\\ [\myskip]
				% \pcfor j \in \range{1}{\abs{\vec{z}}} \pcdo \\ [\myskip]
				(*) \pcif 
          \sum_{j = 1}^{\abs{\vec{z}}} r^{j} \cdot \gone{\sum_{i \in K_j}
          \gamma_j^{i - 1} c_{i} - \sum_{i \in K_j} \gamma_j^{i - 1} s_{i_j}} \bullet \gtwo{1} + \\ [\myskip] 
          \pcind \sum_{j = 1}^{\abs{\vec{z}}} r^{j} z_j o_j
          \bullet \gtwo{1} \neq \gone{\sum_{j = 1}^{\abs{\vec{z}}} r^{j} o_j }
          \bullet \gtwo{\chi} \pcthen  \\
					\pcind \pcreturn 0\\ [\myskip]
          \fbox{
            \begin{minipage}{7cm}
            (**) $\pcif $
              $\sum_{j = 1}^{\abs{\vec{z}}} r^j \cdot (\sum_{i \in K_j}
              \gamma_j^{i - 1} \p{f}_{i} (X) - \sum_{i \in K_j} \gamma_j^{i - 1} s_{i_j}) + $\\ [\myskip] 
              $\pcind \sum_{j = 1}^{\abs{\vec{z}}} r^{j} z_j \p{o}_j (X)
               \neq \sum_{j = 1}^{\abs{\vec{z}}} r^{j} \p{o}_j (X)
              \cdot X \pcthen $ \\
              $\pcind \pcreturn 0$
            \end{minipage}
          }\\[\myskip]
					\pcreturn 1.\\
			}

      \pchspace
      
      \procedure{$\verify(\srs, \gone{\vec{c}}, \vec{z}, \vec{s}, \gone{\p{o}(\chi)},\aux_0)$}
			{
        \vec{\gamma} \gets \ro (g_0( \vec{z}, \vec{s}, \gone{\vec{c}}, \aux_0))\\[\myskip]
				\pcfor j \in \range{1}{\abs{\vec{z}}} \pcdo \\ [\myskip]
				\pcind \pcif 
          \gone{\sum_{i \in K_j}
          \gamma_j^{i - 1} c_{i} - \sum_{i \in K_j} \gamma_j^{i - i} s_{i, j}} \bullet
          \gtwo{1} + \\ [\myskip] \pcind  z_j
          o_j
          \bullet \gtwo{1} \neq \gone{o_j}
          \bullet \gtwo{\chi} \pcthen  \\
					\pcind \pcreturn 0\\ [\myskip]
        \fbox{
          \begin{minipage}{6cm}
          $\pcind \pcif $
          $\sum_{i \in K_j} \gamma_j^{i - 1} \p{f}_{i} (X) - \sum_{i \in K_j} \gamma_j^{i - i} s_{i, j} + $\\ [\myskip] 
          $\pcind  z_j \p{o}_j (X) \neq \p{o}_j (X) X \pcthen \pcreturn 0$
        \end{minipage}
        }\\ [\myskip]
					\pcreturn 1.
			}
    \end{pchstack}
	\end{pcvstack}
\end{minipage}
  }}
	\caption{$\PCOMp$ polynomial commitment scheme. Here $\abs{\vec{z}} = l$ is the number of evaluation points, the number of committed polynomials is $m$, $K_j$ is the set of polynomials that was evaluated at point $z_j$. Functions $g_0$ and $g_1$ are injective and specific to the context in which the polynomial commitment is used. (In our case, functions $g_0$ and $g_1$ are produce partial transcripts of the proof that utilizes the commitment scheme, $\aux$ contains all additional information that is needed by the functions.)
  In the boxes we describe values returned or equality computed in the ideal protocol where the verifier checks equalities on the polynomials instead of their evaluations. For algorithm $\pcalgostyle{Alg}$ we denote its ideal variant by $\pcalgostyle{Alg}'$.}
	\label{fig:pcomp}
  \end{figure}

% \begin{figure}[h!]
% \centering
% 	\begin{pcvstack}[center,boxed]
% 		\begin{pchstack}
% 			\procedure{$\kgen(\secparam, \maxdeg)$} {
% 				\alpha, \chi \sample \FF^2_p \\ [\myskip]
% 				\pcreturn \gone{\smallset{\chi^i}_{i = -\multconstr}^{\multconstr},
%           \smallset{\alpha \chi^i}_{i = -\multconstr, i \neq
%             0}^{\multconstr}},\\
%         \pcind \gtwo{\smallset{\chi^i, \alpha \chi^i}_{i =
%             -\multconstr}^{\multconstr}}, \gtar{\alpha}\\
% 				%\markulf{03.11.2020}{} \\
% 			%	\hphantom{\pcind \p{o}_i(X) \gets \sum_{j = 1}^{t_i} \gamma_i^{j - 1} \frac{\p{f}_{i,j}(X) - \p{f}_{i, j}(z_i)}{X - z_i}}
% 				\hphantom{\hspace*{5.5cm}}
% 		}
%
% 			\pchspace
%
% 			\procedure{$\com(\srs, \maxconst, \p{f}(X))$} {
% 				\p{c}(X) \gets \alpha \cdot X^{\dconst - \maxconst} \p{f}(X) \\ [\myskip]
% 				\pcreturn \gone{c} = \gone{\p{c}(\chi)}\\ [\myskip]
% 				\hphantom{\pcind \pcif \sum_{i = 1}^{\abs{\vec{z}}} r_i \cdot
%           \gone{\sum_{j = 1}^{t_j} \gamma_i^{j - 1} c_{i, j} - \sum_{j = 1}^{t_j}
%             s_{i, j}} \bullet \gtwo{1} + } }
% 		\end{pchstack}
% 		% \pcvspace
%
% 		\begin{pchstack}
% 			\procedure{$\open(\srs, z, s, f(X))$}
% 			{
% 				\p{o}(X) \gets \frac{\p{f}(X) - \p{f}(z)}{X - z}\\ [\myskip]
% 				\pcreturn \gone{\p{o}(\chi)}\\ [\myskip]
% 				\hphantom{\hspace*{5.5cm}}
% 			}
%
% 			\pchspace
%
% 			\procedure{$\verify(\srs, \maxconst, \gone{c}, z, s, \gone{\p{o}(\chi)})$}
%       {
%         \pcif \gone{\p{o}(\chi)} \bullet \gtwo{\alpha \chi} + \gone{s - z
%         \p{o}(\chi)} \bullet \gtwo{\alpha} = \\ [\myskip] \pcind \gone{c}
%         \bullet \gtwo{\chi^{- \dconst + \maxconst}} \pcthen  \pcreturn 1\\
%         [\myskip]
%         \rlap{\pcelse \pcreturn 0.} \hphantom{\pcind \pcif \sum_{i =
%             1}^{\abs{\vec{z}}} r_i \cdot \gone{\sum_{j = 1}^{t_j} \gamma_i^{j -
%               1} c_{i, j} - \sum{j = 1}^{t_j} s_{i, j}} \bullet \gtwo{1} + } }
% 		\end{pchstack}
% 	\end{pcvstack}
%
% 	\caption{$\PCOMs$ polynomial commitment scheme.}
% 	\label{fig:pcoms}
% \end{figure}

\subsection{Unique Opening Property of $\PCOMp$}
\label{sec:uop}
Now, we show that the batched variant of the KZG polynomial
commitment scheme that is used in \plonk{} and $\marlin{}$, has the unique opening property.
%\markulf{30.4}{Changed this to $\PCOMp$, hope that's correct. Could simplfy this, given that the body will be only about Plonk?}

\begin{lemma}
\label{lem:pcomp_op}
Let $\PCOMp = (\kgen, \com, \open, \verifyb)$ be a batched version of a KZG polynomial commitment,
cf.~\cref{fig:pcomp}, that commits to $m$ polynomials of degree up to $\maxdeg$. Let $\vec{z} = (z_0, \ldots, z_{l - 1}) \in \FF_p^l$ be the points the polynomials are  evaluated at, $k_i \in \NN$ be the number of the committed polynomials to be evaluated at $z_i$, and $K_i$ be the set of indices of these polynomials. Let $\vec{s_{K_i}} \in \FF_p^{k_i}$ the evaluations of polynomials at $z_i$, and $\vec{o} = (o_0, \ldots, o_{l - 1}) \in \FF_p^l$ be the commitment openings. We show that the probability an algebraic adversary $\adv$, who can made up to $q$ random oracle queries, opens the same vector of commitments in two different ways is at most $\epsop(\secpar)$, for $\epsop(\secpar) \leq l \cdot  \epsudlog(\secpar) + \infrac{1}{p}$, where $\epsudlog(\secpar)$ is security of the $(\maxdeg, 1)$-$\udlog$ assumption and $p$ is the field used in $\PCOMp$.
\end{lemma}
\begin{proof}
 
  The proof goes by game hops. In the first game the adversary wins if it presents two accepting openings of a vector of polynomials. Then, we restrict the winning condition and require that the adversary also makes the idealized batched verifier to accept the proof. In the next game, we abort if the idealized verifier rejects a proof for one of the evaluation point. 

  \ncase{Game 0} In this game the adversary wins if it provides two different openings for a vector of polynomial commitments and their evaluations that are accepting by $\verifyb$.

  \ncase{Game 1} This game is identical to Game 0 except it is additionally aborted if the commitment opening are not accepting by $\verifyb'$.

  \ncase{Game 0 to Game 1} %
  Any discrepancy between the idealised verifier rejection and real verifier acceptance allows one to break the (updatable) discrete logarithm problem.  The reduction $\rdvdulog$ proceeds as follows. It answers $\adv$'s queries for SRS updates according to the answers it receives from its udlog update oracle. When $\adv$ finalizes an SRS, $\rdvdulog$ finalizes the corresponding udlog challenge $(\gone{1, \ldots, {\chi'}^{\maxdeg}}, \gtwo{1})$. We consider verification equation $(**)$ as a polynomial in $X$ and the verification equation $(*)$ as it's evaluation at $\chi'$. Consider an opening such that verification equation (**), cf.~\cref{fig:pcomp}, does not hold, i.e.~(**) is not a zero polynomial, but (*) does, i.e.~(**) zeroes at $\chi'$. Since $\adv$ is algebraic, all proof elements are extended by their representation as a combination of the input $\GRP_1$-elements. Therefore, all coefficients of the verification equation polynomial related to (**) are known. Now, $\rdvdulog$ computes the roots of (**), finds $\chi'$ among them, and returns $\chi'$. Hence the probability that the adversary wins in Game 1, but does not win in Game 0 is upper-bounded by $\epsudlog (\secpar)$.

  \ncase{Game 2} This game is identical to Game 1 except it is additionally aborted if one of the opening is not accepting by an idealised verifier $\verify'$.

  \ncase{Game 1 to Game 2}
  The ideal verifier checks whether the following equality, for $\smallset{\gamma_j}_{j = 1}^{l}, r$
  picked at random, holds:
  \begin{multline}
    \label{eq:ver_eq_poly}
    \sum_{j = 0}^{l - 1} r^{j}\left(\sum_{i \in K_j} \gamma_{j}^{i - 1} \cdot \p{f}_i(X) - \sum_{i \in K_j} \gamma_{j}^{i - 1} \cdot {s_i}_j \right) 
    \equiv \sum_{j = 0}^{l - 1} r^{j} \p{o_{j}}(X)(X - z_j).
  \end{multline}
  Since $r$ has been picked as a random oracle output, probability that
  \cref{eq:ver_eq_poly} holds while for some $j \in \range{0}{l - 1}$
  \[
    r^j \left(\sum_{i \in K_j} \gamma^{i - 1} \cdot \p{f}_i(X) - \sum_{i \in K_j}
    \gamma^{i - 1} \cdot {s_i}_j \right) \not\equiv r^j \p{o_j}(X)(X - z_j)
  \]
  is $\infrac{q}{p}$~cf.~\cite{EPRINT:GabWilCio19}. 
  When \(
    r^j \left(\sum_{i \in K_j} \gamma^{i - 1} \cdot \p{f}_i(X) - \sum_{i \in K_j}
    \gamma^{i - 1} \cdot {s_i}_j \right) = r^j \p{o_j}(X)(X - z_j)
  \)
  holds, polynomial $\p{o_j}(X)$ is uniquely determined from the uniqueness of
  polynomial composition. 

  \ncase{Conclusion} %
  We note that the idealised verifier $\vereq(X)$ does not accept two different openings of a correct evaluation. Hence the probability that the adversary wins Game 2 is $0$ and the probability that the adversary wins in Game 0 is upper-bounded by \(\epsudlog (\secpar) + \frac{q}{p}\).
    \qed
\end{proof}